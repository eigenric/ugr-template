%%%%%%%%%%%%%%%%%%%%%%%%%%%%%%%%%%%%%%%%%%%%%%%%%%%%%%%%%%%%%%%%%%%%%%%%%%%%%%%%%%%%%%%%%%%
%%
%% The updated version of this document should be downloaded from
%%      https://github.com/jp-um/university_of_malta_LaTeX_dissertation_template
%%
%% In case of any difficulties please contact Dr JP Ebejer on jean.p.ebejer@um.edu.mt
%%
%%%%%%%%%%%%%%%%%%%%%%%%%%%%%%%%%%%%%%%%%%%%%%%%%%%%%%%%%%%%%%%%%%%%%%%%%%%%%%%%%%%%%%%%%%%

%% Before you embark on this quest you should probably read some of:
%% Deadly sins - http://ctan.mirror.garr.it/mirrors/CTAN/info/l2tabu/english/l2tabuen.pdf
%% Writing a thesis in LaTeX - http://tug.org/pracjourn/2008-1/mori/mori.pdf

\RequirePackage[l2tabu, orthodox]{nag} % tells you of any bad LaTeX usage
                                       % must be first thing in class (with the exception of comments)

%% There is one option you should define; oneside or twoside
%% Use twoside for your viva docs (examiners hate long docs they need to carry around)
%% and oneside for the final thing you submit to the library.  Note that margins will
%% change accordingly

\documentclass[oneside, spanish]{um}  % custom University of Malta project/dissertation/thesis 

%% **************** (Your) Packages (Start) ******************

% \listfiles % uncomment this to know which packages you are using
              % the list of packages will be in the bottom of the .log file

%% Note that packges may already be loaded from the um (and memoir) classes.
%% Do not add your packages to the template, but rather add them here.

\usepackage{blindtext} %% for some dummy text, remove in your writeup
\usepackage{coffee4}    %% for some fun

%% ***************** (Your) Packages (End) *******************


%% **************** (Your) Data (Start) ******************

\title{Aprendizaje Automático. \\Búsqueda Iterativa de \\ Óptimos y Regresión Lineal}  % use \\ here otherwise you get a justified title
                                                    % note capitalization of the title (only common 
                                                    % words in lower case)
\tagline{Descendente (Estocástico), Pseudoinversa de Moore-Penrose, Características No Lineales}                     % tag line
\author{Ricardo Ruiz Fernández de Alba}                        
\department{DECSAI}                  % your department (e.g. Artifical Intelligence)
\faculty{Escuela Técnica Ingeniería Informática y Matemáticas}                      % your faculty (e.g. ICT)
\degree{Doble Grado en Ingeniería Informática y Matemáticas}                      % the degree you are reading
\university{Universidad de Granada}
                                                    % note the \ after the dot, so not to consider it a fullstop
\doctype{dissertation}                              % the type of document (fyp, dissertation, thesis)
\degreedate{Abril, 2022}                        % when did you submit -- officially after your corrections !
%%\subjectcode{ICS5200}                               % the study unit-code (currently not used)

%% ***************** (Your) Data (End) *******************


%% ******** (Your) Document Settings (Start) *************

% You should have an images directory in every chapX subdir
% NOTE:  Trailing / for subdirs is required.
\graphicspath{{./images/}{./chap1/images/}{./chap2/images}}   % Paths where to look for images, if defined "images" must always be there as it holds the images in-use by the template.

\makeindex

%% ********* (Your) Document Settings (End) **************

% DOCTOR'S (JP) ORDERS: MAKE SURE TO READ MY TWO BLOG ENTRIES WITH
% CONTENT AND LaTeX TIPS FOR YOUR WRITE-UP.  THESE ARE BASED ON  
% EXAMINER'S FEEDBACK
%
% URLS:
% https://bitsilla.com/blog/2019/03/content-tips-for-your-dissertation-or-project-write-up/
% https://bitsilla.com/blog/2019/01/latex-tips-for-your-dissertation-or-project-write-up/

% end the preamble and start the document

\let\cleardoublepage=\clearpage

\begin{document}
\selectlanguage{spanish}
\frontmatter 
\maketitle

\tableofcontents

%% Note: always use \input as you cannot nest \includes (amongst other things)
\pagestyle{umpage}
\mainmatter 
    \chapter{Ejercicio sobre la búsqueda iterativa de óptimos}

\section{Descenso de gradiente}

El algoritmo de descenso de gradiente es una técnica general para minimizar
funciones diferenciables. Es un algoritmo iterativo basado en el cálculo del
gradiente de la función, y la actualización de cierto vector de pesos.

Formalmente, sea $w(0) \in \mathbb{R}^d$ un punto inicial, y 
$E : \mathbb{R}^d \longrightarrow \mathbb R$
una función diferenciable. Sea $\eta \in \mathbb{R}^+$. La actualización del peso en 
la $t$-ésima iteración viene definida por 

\begin{equation*}
  w(t+1) = w(t) - \eta \nabla E_{in}(w(t))
\end{equation*}

donde al parámetro $\eta$ se le conoce como tasa de aprendizaje. El algoritmo se basa
en seguir la dirección opuesta al vector gradiente pues se maximiza el decrecimiento de
la función en cada iteración. La convergencia no está garantizada en general y dependerá
de la tasa de aprendizaje escogida así como de las características de la función. 

Formalmente, esto se deduce de la expansión de Taylor de primer orden de la función:
\cite{LFD}

\begin{equation}
\begin{aligned}
\Delta E_{in} = E_{in}(w(0) + \eta \hat{v}) - E_{in}(w(0)) \\
= \eta \nabla E_{in}(w(0))^T \hat{v} + O(\eta^2) \\
= \geq - \eta \lVert \nabla E_{in}(w(0)) \rVert
\end{aligned}
\end{equation}

con $\hat{v}$ vector unitario. La igualdad se da si y sólo si

\begin{equation*}
\hat{v} = - \frac{\nabla E_{in}(w(0))}{\lVert \nabla E_{in}(w(0)) \rVert}
\end{equation*}

Una tasa de aprendizaje (demasiado) pequeña es ineficiente lejos del mínimo local. 
Por otro lado, si es demasiado grande, las oscilaciones pueden afectan a la convergencia.
Por ello, optamos por una tasa variable proporcional a la norma del gradiente.

\begin{equation*}
  \eta_t = \eta \lVert \nabla E_{in} \rVert
\end{equation*}

De esta manera, se realizan grandes pasos lejos del mínimo y pasos de menor tamaño
cerca del mínimo (donde la norma del gradiente es menor).
Además, $\eta_t$ cancela con el denominador de $\hat{v}$ definido anteriormente lo que
se sintetiza redefiniendo $\eta = \eta_t$ y hallando el opuesto del vector gradiente 
pero sin normalizar.

Como la convergencia no está garantizada en tiempo finito, necesitamos una condición de
parada como que la función a minimizar quede por debajo de cierto umbral de
error (i.e $E_{in} \leq \epsilon$) o simplemente un número máximo de iteraciones.
Combinamos ambas en la implementación ofrecida an el archivo de código:

\section{Ejercicio 1}

\textbf{Implementar el algoritmo de gradiente descendente}

Analizamos la definición de la función descrita en el código:

\begin{minted}{python}
def gradient_descent(w_ini, lr, grad_fun, fun, epsilon, max_iters, 
                     stop_cond,  hist=False):
\end{minted}

Sobre la cabecera:
\begin{itemize}
  \item \mintinline{python}{w_ini}: vector de pesos inicial
  \item \mintinline{python}{grad_fun}: gradiente de la función a minimizar
  \item \mintinline{python}{fun}: función a minimizar
  \item \mintinline{python}{epsilon}: cota de error para parar 
  \item \mintinline{python}{max_iters}: número máximo de iteraciones para parar
  \item \mintinline{python}{stop_cond(it, w, epsilon, max_iters, grad_fun, fun)}
    : función que indica cuando parar el algoritmo
    \subitem \mintinline{python}{stop_cond_maxIter}: para cuando \mintinline{python}{it <= max_iters}
    \subitem \mintinline{python}{stop_cond_error}: para cuando \mintinline{python}{fun(w[0], w[1]) \leq epsilon}
  \item \mintinline{python}{hist}: True si se desea devolver el histórico para visualización
\end{itemize}


\section{Ejercicio 2.}
Considerar la función $E(u, v) = \left( u v \cdot e^{ -u^2 -v^2 } \right) ^2$
Usar gradiente descendente para encontrar un mínimo de esta función, comenzando
desde el punto $(u, v) = (0.5, -0.5)$ y usando una tasa de aprendizaje $\eta = 0,1$.

\subsection{Cálculo análitico de \texorpdfstring{$\nabla E(u, v)$}{Lg}}

Hallamos la primera derivada parcial: 

\begin{equation}
\begin{aligned}
  \frac{\partial{E}}{\partial u}(u,v) = 2uve^{-u^2 - v^2} \cdot v \left[ e^{-u^2 - v^2} + u\cdot e^{-u^2 - v^2} \cdot 2(-u) \right] = \\
  = 2uv^2e^{2(-u^2-v^2)} - 4u^3v^2e^{2(-u^2-v^2)} = -2u (2u^2 - 1) v^2 e^{-2(u^2+v^2)}
\end{aligned}
\end{equation}

Y análogamente, $\frac{\partial{E}}{\partial u}(u,v) = -2v (2v^2 - 1) u^2 e^{-2(u^2+v^2)}$. Luego, 
\begin{equation}
  \begin{aligned}
    \nabla{E}(u,v) = \begin{pmatrix}
           \frac{\partial E}{\partial u}(u,v) \\
           \frac{\partial E}{\partial v}(u,v) 
    \end{pmatrix}  =  \begin{pmatrix}
          -2u (2u^2 - 1) v^2 e^{-2(u^2+v^2)} \\
          -2v (2v^2 - 1) u^2 e^{-2(u^2+v^2)}
    \end{pmatrix}
  \end{aligned}
\end{equation}


\subsection{Número de iteraciones con cota de error \texorpdfstring{$10^{-8}$}{Lg}}
\subsection{Coordenadas obtenidas con cota de error \texorpdfstring{$10^{-8}$}{Lg}}

Ejecutamos el algoritmo anteriormente descrito con los siguiente parámetros:

\begin{minted}{python}
eta = 0.1 
error2get = 1e-8
initial_point = np.array([0.5,-0.5])

w, it = gradient_descent(initial_point, eta, gradE, E, 
                         error2get, maxIter, stop_cond_error)
\end{minted}

La condición de parada fijada es que la función de error alcance un valor
menor o igual a $10^{-8}$, que se indica en \mintinline{python}{stop_cond_error}

\begin{itemize}
\item Numero de iteraciones: $25117$
\item Coordenadas obtenidas: $(0.0100, -0.0100)$
\end{itemize}


En la siguiente figura visualizamos el desplazamiento hacia el mínimo: 

\begin{figure}[H]
\centering
\includegraphics[width=0.5\textwidth]{Figure_1.png}
\caption{Descenso de gradiente sobre E}
\end{figure}

\section{Ejercicio 3}

Consideramos la función  $f(x,y) = x^2 + 2y^2 + 2 \sin (2 \pi x) \sin (\pi y)$.
\textbf{Aplicar gradiente descendente para minimizar $f$}

\subsection{Dependencia de la tasa de aprendizaje}

Usar como punto inicial $(x_0 = -1, y_0 = 1)$, tasa de aprendizaje $\eta = 0,01$
y un máximo de $50$ iteraciones. Generar un gráfico de cómo desciende el valor
de la función con las iteraciones. Repetir el experimento pero usando 
$\eta =0,1$, comentar las diferencias y su dependencia de $\eta$.  

En primer lugar, calculamos analíticamente el gradiente de $f$.

\begin{equation}
\begin{aligned}
  \nabla f(x,y) = \begin{pmatrix}
  \frac{\partial }{\partial x} f(x,y) \\
  \frac{\partial }{\partial y} f(x,y)
  \end{pmatrix} = \begin{pmatrix}
   2x + 4\pi sen(\pi y)cos(2\pi x) \\
   4y + 2\pi sen(\pi x)cos(\pi y)
  \end{pmatrix}
\end{aligned}
\end{equation}

En este caso la condición de parada es el máximo de iteraciones, por lo que
que ejecutamos el algoritmo con la función de condición de parada
\mintinline{python}{stop_cond_maxIter}

\begin{minted}{python}
eta = 0.01 
maxIter = 50
initial_point = np.array([-1, 1])

ws, it = gradient_descent(initial_point, eta, gradf, f, None, maxIter,
                         stop_cond_maxIter, hist=True)

\end{minted}

Así, para $\eta=0.01$ y un máximo de $50$ iteraciones, las coordinadas finales
son $(-1.2178, 0.4134)$ donde $f$ toma el valor $-0.0623$

\begin{figure}[H]
\centering
\includegraphics[width=0.6\textwidth]{Figure_2.png}
\caption{Descenso del gradiente sobre f con $\eta=0.01$}
\end{figure}

De igual manera para $\eta=0.1$ las coordenadas finales obtenidas son 
$(-0.1155, 0.1610)$ y visualizamos el descenso en la siguiente figura:

\begin{figure}[H]
\centering
\includegraphics[scale=0.75]{Figure_3.png}
\caption{Descenso del gradiente sobre f con $\eta=0.1$}
\end{figure}

Como se observa en la figura, la tasa de aprendizaje es demasiado alta,
provocando oscilaciones que saltan el mínimo. De hecho, una inspección del
vector de puntos \mintinline{python}{ws} revela que en la $43$-ésima iteración
alcanza el mínimo valor de las $50$ iteraciones, pero que luego sigue oscilando y
termina en la última iteración con un valor mayor.

La comparación entre las oscilaciones con $\eta=0.1$ y el descenso adecuado con 
$\eta=0.01$ quedan reflejadas en la siguiente figura.

\begin{figure}[H]
  \centering
  \includegraphics[width=0.6\textwidth]{Figure_4.png}
  \caption{Comparación de $f(x,y)$ para $\eta=0.01$ y $\eta=0.1$}
\end{figure}

% Análogamente, si elegimos un valor menor, i.e $\eta = 0.05$, obtendremos
% un descenso más lento (más ineficiente) para el mismo número de iteraciones ($50$)
% en comparación con $\eta = 0.01$, como podemos observar en la siguiente gráfica:

% \begin{figure}[H]
%   \centering
%  \includegraphics[scale=0.6]{Figure_5.png}
%  \caption{Comparación de $f(x,y)$ para $\eta=0.01$ y $\eta=0.05$}
% \end{figure}


\subsection{Tabla de mínimos según punto inicial y tasa de aprendizaje}

Obtener el valor mínimo y los valores de las variables $(x, y)$ en donde se
alcanzan cuando el punto de inicio se fija en: $(-0.5, -0.5)$, $(1, 1)$, $(2.1,-2.1)$,
$(-3, 3)$, $(-2, 2)$.
Generar una tabla con los valores obtenidos, empleando el máximo número de
iteraciones ($50$) y las tasas de aprendizaje del anterior apartado ($\eta=0.01$ y 
$\eta = 0.1$). Comentar la dependencia del punto inicial.

\begin{table}[!ht]
    \caption {Valor mínimo para $\eta = 0.1$} \label{tab:title} 
    \centering
    \begin{tabular}{lll}
    \toprule
        Punto inicial & Coordenadas mínimo & Valor mínimo f(x,y) \\ \midrule
        (-0.5, -0.5) & (0.2848,-0.5553) & -1.2250 \\
        (1, 1) & (0.3018,-0.4257) & -1.3902 \\ 
        (2.1, -2.1) & (-0.2320,0.2831) & -1.3293 \\
        (-3, 3) & (0.1816,-0.3152) & -1.2883 \\
        (-2, 2) & (0.2428,-0.3121) & -1.4061 \\ \bottomrule
    \end{tabular}
\end{table}

\begin{table}[!ht]
    \caption {Valor mínimo para $\eta = 0.01$} \label{tab:title2} 
    \centering
    \begin{tabular}{lll}
    \toprule
        Punto inicial & Coordenadas mínimo & Valor mínimo f(x,y) \\ \midrule
        (-0.5, -0.5) & (-0.7308,-0.4144) & -1.0366 \\ 
        (1, 1) & (0.7308,0.4144) & -1.0366 \\ 
        (2.1, -2.1) & (1.6651,-1.1728) & 4.6338 \\ 
        (-3, 3) & (-2.1888,0.5868) & 3.6942 \\ 
        (-2, 2) & (-1.6643,1.1713) & 4.6337 \\ \bottomrule
    \end{tabular}
\end{table}

Es claro a partir de estos datos que el algoritmo de gradiente Descendente tiende
hacia un mínimo local que dependerá del punto inicial. En general, en este caso,
la tasa de aprendizaje mayor $\eta=0.1$ proporciona un valor mínimo más regular
(en torno a $-1.25$) que la tasa $\eta=0.01$. 

\section{Ejercicio 4}

\textbf{¿Cuál sería su conclusión sobre la verdadera dificultad de encontrar el mínimo
global de una función arbitraria?}

A partir del experimento anterior, confirmamos empíricamente que el mínimo local
que encuentra el algoritmo de gradiente Descendente depende fuertemente tanto
del punto inicial como de la tasa de aprendizaje. No podemos garantizar que este
mínimo sea global a no ser que la función sea convexa. Esto último sucede en los casos
de la función de error cuadrático medio de la regresión lineal así como en el 
error de entropía cruzada de la regresión logística. 
Esto significa que al minimizar estas medidas de error convexas con este
algoritmo, no se estancará en un mínimo local.
    \chapter{Regresión Lineal}

\section{Algoritmo de la Pseudo-inversa}
El algoritmo de la Pseudo-inversa o mínimos cuadrados ordinarios consiste
en minimizar el error cuadrático entre la estimación $h(x) = w^T x$ y
el vector objetivo $y$

\begin{equation}
E_{in}(h) = \frac{1}{N} \sum_{n=1}^N (h(x_n) - y_n)^2 = \frac{1}{N} \Vert Xw - y\Vert^2
\end{equation}

Basta encontrar $w$ que anule $\nabla E_{in}(w)$. Esto se verifica cuando
$w_{lin} = X^\dagger y$ \cite{LFD}. Donde $X^\dagger = (X^T X)^{-1} X^T$ es la
\textbf{matriz pseudo-inversa} de $X$

Como el cálculo de la inversa $(X^T X)^{-1}$ puede ser costoso, aplicamos
previamente la descomposición en valores singulares a $X$. Es decir, escribimos
$X = U D V^T $ con $U \in \mathbb{R}^{N \times N}$, $V \in \mathbb{R}^{(d+1) \times (d+1)}$
matrices ortogonales y $D \in \mathbb{R}^{N \times (d+1)}$ matriz diagonal rectangular.
Entonces $X^T X = V D^T D V^T$ y la pseudo-inversa se calculará como sigue:

\begin{equation}
\begin{aligned}
X^\dagger = (X^T X)^{-1} X^T = (V D^T D V^T)^{-1} V D^T U^T = \\
= V (D^T D)^{-1} V^T V D^T U^T= V (D^T D)^{-1} D^T U^T = V D^\dagger U^T
\end{aligned}
\end{equation}


\section{Gradiente Descendente Estocástico}

El algoritmo de gradiente descendente estocástico (SGD) es una versión secuencial
del algoritmo de gradiente descendente descrito en el primer capítulo.
En esta versión, usamos sólo una parte de la muestra para calcular el
gradiente. Para ello, dividimos la muestra aleatoriamente en
\textbf{mini-batches}. Recorreremos esta secuencia de lotes, actualizando los
pesos en cada mini-batch de manera que sólo uno participa en cada actualización.

Formalmente, para cada mini-batch $B_i$ 

\begin{equation*}
  \frac{\partial E_{in}(w)}{\partial w_j} = \frac{2}{M} \sum_{n \in B_i} x_{nj} (h(x_n) - y_n)
\end{equation*}

\section{Ejercicio 1}

Este ejercicio ajusta modelos de regresión a vectores de características
extraídos a partir de imágenes de dígitos manuscritos. En particular, se
extraen dos características concretas que miden el valor medio del nivel de
gris y la simetría del dígito respecto de su eje vertical. Solo se
seleccionaran para este ejercicio las imágenes de los números 1 y 5.

\textbf{Estimar un modelo de regresión lineal}, a partir de los datos proporcionados
por los vectores de características dados, usando tanto el algoritmo de la
pseudo-inversa como el gradiente descendente estocástico (SGD). Las etiquetas
serán {-1,1}, una por cada vector de cada uno de los números. Pintar las
soluciones obtenidas junto con los datos usados en el ajuste. \textbf{Valorar la
bondad del resultado usando $E_{in}$ y $E_{out}$} (para $E_{out}$ calcular las
predicciones usando los datos del fichero de test).

\hfill \break

En el caso de la pseudo-inversa, se ha implementado usando la descomposición en
valores singulares anteriormente descrita en \mintinline{python}{def pseudoinverse(x):}
y se ha aplicado al problema de regresión con cierto vector objetivo $y$ en 
\mintinline{python}{def regresion_pinv(x, y)}.

En cuanto al algoritmo de gradiente descendente estocástico (SGD), se ha implementado
en su versión \textit{mini-batch} de forma general mediante una condición genérica de
parada, como se hizo con el no estocástico. Esta se ha particularizado para el caso 
de un batch de tamaño $32$ y una condición de parada por máximo de iteraciones en 

\begin{minted}{python}
def sgd_maxIter(x, y, lr, max_iters, epsilon=None, 
                batch_size=32, hist=False):
\end{minted}

También existe otra versión (\mintinline{python}{sgd_maxIter_error()}) con
condición de parada mixta por máximo de iteraciones o cota de error.

Tras cierta experimentación con los parámetros, se ha decidido exponer la comparación entre SGD con
un $500$ iteraciones, SGD con $20000$ iteraciones y el algoritmo de la
pseudoinversa. En particular, se ha escogido una tasa de aprendizaje 
$\eta = 0.01$ por la estabilidad proporcionada.

\begin{figure}[H]
\centering
\includegraphics[scale=0.6]{Figure_6.png}
\caption{Descenso del gradiente sobre f con $\eta=0.01$}
\end{figure}

\begin{table}[!ht]
    \centering
    \begin{tabular}{llll} 
    \toprule
        Algoritmo & Iteraciones & $E_{in}$ & $E_{out}$ \\ \midrule
        SGD (32)  & 500 & 0.083469 & 0.132706 \\ 
        SGD (32) & 20000 & 0.080445 & 0.135549 \\ 
        Pseudoinversa  & ---- &  0.079187 & 0.130954 \\ \bottomrule
    \end{tabular}
\end{table}

Se observa que $E_{in}$ producido por la pseudoinversa es menor que el del
Gradiente Descendente Estocástico. Esto se debe a que el algoritmo de la
pseudoinversa se basa en la minimización de este error. Ahora bien, a partir
cierto tamaño de datos, este algoritmo se puede volver costoso por las operaciones
matriciales involucradas (si bien reducimos esta complejidad mediante la descomposición
en valores singulares). Además, el componente aleatorio de SGD permite escapar de mínimos
locales al contrario del algoritmo normal tratado en el capítulo 1.

En cuanto a las $E_{out}$ originados por las predicciones en test, vemos que son
mayores que los $E_{in}$. Esto sucede porque el algoritmo no se adapta a los datos
de test igual que a los datos con los que ha sido entrenado.

Finalmente, podemos notar a partir de la gráfica junto con experimentaciones con mayores
valores de iteraciones que el algoritmo de gradiente descendente estocástico
genera una recta de regresión que se acerca a la producida por el algoritmo de
la pseudoinversa conforme se aumentamos el número de iteraciones.


\section{Ejercicio 2}

En este apartado exploramos cómo se transforman los errores $E_{in}$ y $E_{out}$
cuando aumentamos la complejidad del modelo lineal usado. Ahora hacemos uso de
la función \mintinline{python}{simula_unif(N, 2, size)} que nos devuelve $N$
coordenadas 2D de puntos uniformemente muestreados dentro del cuadrado definido
por  $[-size,size] \times [-size, size]$. Se debe realizar el siguiente experimento:

\subsection{Generar muestra de entrenamiento}

\textbf{Generar una muestra de entrenamiento de $N = 1000$ puntos en el cuadrado
$\chi = [-1, 1] \times [-1, 1].$ Pintar el mapa de puntos 2D.}

\begin{figure}[H]
\centering
\includegraphics[scale=0.6]{Figure_7.png}
\caption{Muestra aleatoria uniforme con $N = 1000$ en $\chi=[-1, 1] \times [-1, 1]$}
\end{figure}


\subsection{Etiquetado de la muestra}

Consideremos la función $f(x_1, x_2) = sign\left( (x_1 - 0,2)^2 + x_2^2 - 0,6 \right)$ 
que usaremos para asignar una etiqueta a cada punto de la muestra anterior.
Introducimos ruido sobre las etiquetas cambiando el signo de un 10\%
de las mismas elegido aleatoriamente. \textbf{Pintar el mapa de etiquetas obtenido.}

\begin{figure}[H]
\centering
\includegraphics[scale=0.6]{Figure_8.png}
\caption{Etiquetado de la muestra según $f$ con ruido de $10\%$}
\end{figure}

Una vez generado el vector de etiquetas $y$ aplicando $f$ a la muestra, se ha añadido 
el ruido con \mintinline{python}{y[noise] = -y[noise]} siendo \mintinline{python}{noise}
un vector aleatorio uniforme de $0.1 \cdot N$ índices entre $0$ y $N$. 


\subsection{Modelo de regresión Lineal.}

Usando como vector de características $(1, x_1, x_2)$, ajustar un modelo de
regresion lineal al conjunto de datos generado y estimar los pesos w. 
\textbf{Estimar el error de ajuste $E_{in}$ usando SGD.}

Usando SGD con $\eta = 0.01$ y $200$ iteraciones, se ha obtenido:

\begin{itemize}
\item $E_{in} = 0.9293368608301051$
\item $w = [0.02527404, -0.35022946, -0.01136838]$
\end{itemize}

\begin{figure}[H]
\centering
\includegraphics[scale=0.6]{Figure_9.png}
\caption{Modelo de regresión lineal a la muestra generada}
\end{figure}
  
En efecto, el error de ajuste es altísimo debido a que los valores se han
generado aleatoriamente con distribución uniforme y no se acercan a ningún
modelo lineal.

\subsection{Repetir experimento $1000$ veces}

\textbf{Ejecutar todo el experimento definido en los apartado anteriores 1000
veces (generamos 1000 muestras diferentes) y}

\begin{itemize}
  \item \textbf{Calcular el valor medio de los errores $E_{in}$ de las $1000$ muestras.}
  \item \textbf{Generar $1000$ puntos nuevos por cada iteración y calcular con ellos el
  valor de $E_{out}$ en dicha iteración.}
  \item \textbf{Calcular el valor medio de $E_{out}$ en todaslas iteraciones.}
\end{itemize}

Se ha decidido ejecutar por cada repetición, SGD con $200$ iteraciones para
obtener un resultado en no más de $10$ segundos. Los promedios de $E_{in}$ y
$E_{out}$ son respectivamente:

\begin{itemize}
  \item $E_{in}$ promedio: 0.9325394003097736
  \item $E_{out}$ promedio: 0.9357118616492284 
\end{itemize}


\textbf{Valore qué tan bueno considera que es el ajuste con este modelo lineal a la vista de los
valores medios obtenidos de $E_{in}$ y $E_{out}$.}

En efecto, los erores de ajuste obtenidos en promedio para $1000$ iteraciones del
experimento anterior no han producido una mejora significativa. Es claro de las figuras
obtenidas anteriormente, que un modelo lineal no separa etiquetas acumuladas en
circunferencia.

\section{Repetición del experimento con características no lineales}

\textbf{Repetir el mismo experimento anterior pero usando características no
lineales. Ahora usaremos el siguiente vector de características:}

$$
\phi_2(x) = (1, x_1, x_2, x_1 x_2, x_1^2, x_2^2)
$$

\textbf{Ajustar el nuevo modelo de regresión lineal y calcular el nuevo vector de pesos $\hat{w}$.
Calcular los errores promedio de $E_{in}$ y $E_{out}$.}

\hfill \break

Para la implementación, se ha decidido modificar la función \\
\mintinline{python}{generar_muestra_2D_uniforme(N, no_lineal)}
donde el parámetro \mintinline{python}{no_lineal} es un booleano
que indica si agregar las nuevas columnas.

Los errores promedio obtenidos en este caso, han sido los siguientes:

\begin{itemize}
\item $E_{in}$ promedio: $0.7736520275418629$
\item $E_{out}$ promedio: $0.7785544126191773$
\end{itemize}

Y la figura asociada a la transformación es la siguiente:

\begin{figure}[H]
\centering
\includegraphics[scale=0.6]{Figure_10.png}
\caption{Regresión con características no lineales $\Phi(x)$}
\end{figure}

\section{Comparación entre los experimentos}

A la vista de los resultados de los errores promedios $E_{in}$ y $E_{out}$
obtenidos en los dos experimentos, \textbf{¿qué modelo considera que es el más
adecuado? Justifique la respuesta.}

\hfill \break

La figura generada encaja con los resultados. El modelo con características no
lineales aplica una transformación donde la regresión lineal calculada con el
algoritmo de gradiente descendente estocástico con $200$ iteraciones y $\eta=0.01$
arroja unos errores de ajustes menores ($E_{in} \approx 0.7738$ , $E_{out} \approx = 0.778$),
que sin la transformacíón ($E_{in} \approx 0.9320$, $E_{out} \approx 0.9357$).
    \chapter{Método de Newton}

\textbf{This section should include a recipe of what you did (explain what you
have done so if someone wants to reproduce the experiment, they can).  A flow
chart is typically helpful.  Also, make sure to define all software that you
used including version numbers and OS.  Should also include a description of
statistical methods used (if any).\footnote{For more information see:
\url{http://rc.rcjournal.com/content/49/10/1229.short}}}

    %\appendix
    %    \chapter{Media Content}

If the dissertation has a DVD or pendrive attached to it, you will need a
section which explains what is on the media (structure, files, data, etc.).
This could be a table with filename and description.

     % these are just test names as I didn't know what you'd want
    %    \input{appB/appendix_b_main}    
    %    \input{appC/appendix_c_main} 

{\backmatter
    % Bibliography
    \if@openright\cleardoublepage\else\clearpage\fi
    \bibliographystyle{um-plainnat} %% specific plainnat does not show url for articles
    {\footnotesize\bibliography{chap1/gradient_biblio}}
	\printindex
}

\end{document}

%%% The End %%%